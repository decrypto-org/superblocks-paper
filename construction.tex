\section{Interlinks as sets of superblocks}\label{sec.construction}

In superblock-enabled blockchains, the interlink vector stored in each block $B$
contains one pointer per superblock level $\mu$, namely a pointer to the most
recent superblock preceding $B$ of the respective level $\mu$. This
construction, known as an \emph{interlink list}, is realized by inductively
updating the interlink of the previous block, as shown in
Algorithm~\ref{alg.interlink}. Trivially, genesis has an empty interlink.

\import{./}{algorithms/alg.interlink.tex}

Here, we make the novel but simple observation that the interlink structure
constructed in this manner often contains \emph{duplicate pointers}. In fact, as
we will show, most of the interlink pointers are duplicate. Space can be saved
by constructing an \emph{interlink set} instead. This construction is shown in
Algorithm~\ref{alg.interlink-set}. Here, $\epsilon$ denotes the empty sequence
and we address lists using standard Python notation (hence, $\chain[-1]$ denotes
the tip of $\chain$ and $\chain[:-1]$ denotes everything but the tip). The
algorithm returns the exact same data structure as
Algorithm~\ref{alg.interlink}, but with duplicates removed. Note that it can
also be directly adopted to work in an online fashion.

\import{./}{algorithms/alg.interlink-set.tex}

Let $B^n_\mu \in \{0, 1\}$ denote the random variable containing the value of
the $\mu^{\text{th}}$ digit after $n$ turns. Then we have that:

\begin{align*}
\Pr[B^n_\mu = 1] =
\sum_{i = 1}^n 2^{-\mu} \prod_{j = i + 1}^n \sum_{\mu' = 1}^{\mu - 1} 2^{-\mu'} &=\\
\sum_{i = 1}^n 2^{-\mu} \prod_{j = i + 1}^n (1 - 2^{1 - \mu}) &=\\
\sum_{i = 1}^n 2^{-\mu} (1 - 2^{1 -\mu})^{n - (i + 1) + 1} &=\\
\sum_{i = 1}^n 2^{-\mu} (1 - 2^{1 -\mu})^{n - i} &=\\
2^{-\mu} \sum_{i = 1}^n (1 - 2^{1 -\mu})^{n - i} &=\\
\end{align*}
