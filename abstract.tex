\begin{abstract}
  Blocks in proof-of-work blockchains satisfy the proof-of-work equation $H(B)
  \leq T$. If additionally a block satisfies $H(B) \leq T2^{-\mu}$, it is called
  a $\mu$-\emph{superblock}. Superblocks play an important role in the
  construction of compact blockchain proofs which allows the compression of
  proof-of-work blockchains into so-called \emph{proof of proof-of-work}
  certificates. In this work, we measure the distribution of superblocks in the
  Bitcoin blockchain. We find that the superblock distribution within the blockchain
  follows expectation, hence we empirically verify that these blockchains are
  not under previously theoretically posed attacks of \emph{badness}. Given
  these empirical statistics, we discuss efficient ways to store the
  \emph{interlink} data structure, which is a data structure within blocks that
  commits a reference to the most recent preceding $\mu$-superblock for
  every $\mu \in \mathbb{N}$. We observe that, when constructing an interlink by collecting
  superblocks into a Merkle tree, repeated superblock references can be omitted
  with no harm to security. Hence, it is more efficient to store a set of
  superblocks rather than a list. We analytically prove that, in honest
  executions, this simple observation reduces the number of superblock
  references by approximately a half in expectation. We then verify our theoretical
  result by measuring the improvement over existing blockchains in terms of the
  interlink sizes (which we improve by $83\%$) and the sizes of succinct proofs
  of proof-of-work (which we improve by $44\%$).
\end{abstract}
