\begin{abstract}
  Blocks in proof-of-work (PoW) blockchains satisfy the PoW equation $H(B)
  \leq T$. If additionally a block satisfies $H(B) \leq T2^{-\mu}$, it is called
  a $\mu$-\emph{superblock}. Superblocks play an important role in the
  construction of compact blockchain proofs which allows the compression of
  PoW blockchains into so-called \emph{Non-Interactive Proofs of Proof-of-Work} (NIPoPoWs). These certificates are essential for the construction of \emph{superlight}
  clients, which are blockchain wallets that can synchronize exponentially faster than traditional SPV clients.

  In this work, we measure the distribution of superblocks in the Bitcoin
  blockchain. We find that the superblock distribution within the blockchain
  follows expectation, hence we empirically verify that the distribution of superblocks within the Bitcoin blockchain has not been adversarially biased.
  NIPoPoWs require that each block in a blockchain points to a sample of previous blocks in the blockchain. These pointers form a data structure called the \emph{interlink}. We give efficient ways to store the
  interlink data structure. Repeated superblock references within an interlink can be omitted
  with no harm to security. Hence, it is more efficient to store a \emph{set} of
  superblocks rather than a \emph{list}. We show that, in honest
  executions, this simple observation reduces the number of superblock
  references by approximately a half in expectation. We then verify our theoretical
  result by measuring the improvement over existing blockchains in terms of the
  interlink sizes (which we improve by $79\%$) and the sizes of succinct NIPoPoWs (which we improve by $25\%$). As such, we show that deduplication allows superlight clients to synchronize $25\%$ faster.
\end{abstract}
