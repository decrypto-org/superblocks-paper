\section{Superblocks and proofs-of-proofs}

Blocks generated in proof-of-work~\cite{C:DwoNao92} systems must satisfy the
proof-of-work equation $H(B) \leq T$ where $T$ denotes the mining
target~\cite{SP:BMCNKF15} and $B$ denotes the block contents, which is a triplet
including a representation of the application data and metadata, a nonce, and a
reference to the previous block by its hash. The function $H$ is a hash
function, modelled as a random oracle~\cite{CCS:BelRog93}, which outputs
$\kappa$ bits, where $\kappa$ is the security parameter of the protocol and $T <
2^\kappa$. It sometimes happens that some blocks satisfy a stronger version of
the equation~\cite{popow}, namely that $H(B) \leq T2^{-\mu}$ for some $\mu \in
\mathbb{N}$. Such blocks are called $\mu$-superblocks~\cite{nipopows}.
It follows directly from the Random Oracle model that
$\Pr[B \text{ is a } \mu\text{-superblock}|B \text{ valid block}] = 2^{-\mu}$.

The count of superblocks in a chain decreases exponentially as $\mu$ increases.
If a blockchain $\chain$ generated in an honest execution has $|\chain|$ blocks,
it only has $2^{-\mu}|\chain|$ superblocks of level $\mu$ in expectation. Hence,
the total number of levels is $\lg(|\chain|)$ in expectation. It has been
theoretically posed that the distribution of superblocks can be adversarially
biased in so-called ``badness'' attacks~\cite{nipopows} in which an adversary
reduces the density of superblocks of a particular level within a blockchain.
However, the actual distribution of superblocks in currently deployed
blockchains has not been measured. Therefore, it was previously unknown whether
such attacks are taking place in the wild. In this paper, we make empirical
measurements of superblock distributions and observe that they follow the
expectaiton. Hence, we conclude that badness attacks have not occurred in
practice.

For any block $B$,
it is useful to be able to refer to its most recent preceding $\mu$-superblock
for any $\mu \in \mathbb{N}$. In addition, it is useful to include this
reference to the contents of the block to which proof-of-work is being applied
so that the miner proves that she had knownledge of the preceding superblock
when $B$ was generated. For this purpose, it has been
recommended~\cite{nipopows} that for each block $B$, instead of including only a
pointer to the previous block, $\lg(|\chain|)$ pointers will be included, one
for each level $\mu$ pointing to the most recent $\mu$-superblock preceding $B$.
These pointers change the blockchain into a block skiplist \TODO{TODO skiplist
citation}.

These $\lg(|\chain|)$ pointers per block are called the \emph{interlink} data
structure and can either be included in the block header \emph{verbatim}, or
organized into a more compact data structure such as a Merkle tree~\cite{merkle}
\TODO{TODO citation} containing one leaf per superblock level $\mu$.
Proofs-of-inclusion in this Merkle tree are then $\Theta(\lg\lg(|\chain|))$. The
root of this Merkle tree can be included in the block header, replacing the
typical \textsf{previd}. This is done in blockchains adopting interlinking from
genesis or through a hard fork. More commonly, the interlink Merkle tree root
can be included in the block's application data. Thus, it can, for instance,
appear in the coinbase transaction in the case of a soft or velvet
fork~\cite{velvet}, or a specially crafted \emph{velvet transaction} in the case
of a user-activated velvet fork. User-activated velvet forks allow the adoption
of a new rule without requiring miners to upgrade their software or be aware of
the change. In case of a velvet fork, the interlink proof-of-inclusion, which
proves that the superblock pointer was included in the interlink root, must be
accompanied by a transaction proof-of-inclusion, which proves that the interlink
root has been included in the application data committed to the block header
(either in the coinbase transaction, if done by a miner, or in the velvet
transaction, if done by a user).

Superblock pointers can be used to traverse the blockchain from the tip back to
genesis in a manner which skips some unnecessary intermediary blocks and
includes others. The idea is to find a succinct sample of blocks such that it is
a subchain of the blockchain, i.e., each block in the subchain includes a
pointer (either directly through \textsf{previd} or indirectly through the
interlink vector) to its immediately preceding block in the subchain (which
corresponds to an ancestor block in the underlying blockchain). Because each
pointer is proof-of-work committed in the interlink data structure, it suffices
to collect the chosen blocks along with the proofs-of-inclusion. By cleverly
choosing which blocks to collect, a full node can prove to a so-called
\emph{superlite node} that the currently adopted longest blockchain is the
claimed one without presenting the whole blockchain. Hence, instead of
transmitting data linear in the chain size $\Theta(|\chain|)$ as SPV clients do,
it is sufficient to transmit succinct certificates which are only of size
$\Theta(\emph{poly}\log(|\chain|))$. Such certificates are called Non-Interactive
Proofs of Proof-of-Work~\cite{nipopows}. The NIPoPoWs protocol is parameterized
by a security parameter $m$ and requires $2m$ blocks to be included for each
superblock level $\mu$ which contains at least $m$ blocks, except for the
highest level $\mu$ for which only $m$ blocks need to be included. As the number
of levels containing at least $m$ blocks is $\log(|\chain|) - \log(m)$, the
number of blocks included in a NIPoPoW are $\log(|\chain|) - \log(m)$.
